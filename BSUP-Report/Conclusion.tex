\section{Conclusion}

Several of the methods and tools presented in \emph{Project Management for
Information Systems}\cite{caye} were found to be useful in a project like ours,
while others were modified to fit our scale and scope.

Using risk management is an effective way to handle issues before they
manifest, but striking the right balance of how many risks should be handled
can be tricky, and depends on the size of the project. In our case, it would
have been beneficial to address a few more of the recognized risks.

Planning Poker has proven to be useful due to its ease of use, needing
almost no introduction, as well as its proneness to start discussions about the
tasks being estimated, all without taking too much time to perform.
Estimates, no matter what method is used, are hard to get right, especially
in a project where new technologies are used and the team is newly assembled.
Even so, time estimates are both necessary and valuable.

Project planning depends on the correctness of estimates, and are of limited use
if these turn out to be wrong. Some types of plans, like Dependency Diagrams, do
still provide valuable information, as they clarify what activities are on the
\emph{critical path}, and hence will affect the entire project if delayed.

The use of a quality plan has reduced the time needed to perform quality checks
on program parts, freeing up more time to do actual work.

Using Scrum as a light-weight development framework in a small and volatile
project proved very useful, because it allows the project to evolve and change
over time, thus also allowing the requirements of the product to change. Its
agile nature makes it work well with several different estimation methods, and
allows for tasks to be estimated anew, multiple times, limiting the impact of
incorrect estimates.

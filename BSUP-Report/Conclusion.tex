\section{Conclusion}
Several of the methods and tools presented in the book \cite{caye} were found to be useful in a project like this one, while others were not. 
Planning Poker has proven to be were useful due its ease of use, needing almost no introduction, as well as its proneness to start discussions about the tasks being estimated, all without taking very much time to perform.
The use of a quality plan has reduced the time needed to perform quality checks on program parts, freeing up more time to do actual work.
One tool that has proven to be a very useful tool in software development is Scrum because it allows the project to evolve and change over time, thus also allowing the requirements of the product to change. Its agile nature makes it work well with a lot of estimation methods because it allows for tasks to be estimated anew, multiple times, should the previous estimate have been too ambitious. The flexibility of Scrum allows a project group to tailor it to their needs. Being able to run short sprints has been an absolute necessity as the Singaporean team had a lot of spontaneous requirements that needed to be accommodated and incorporated into the requirements as soon as possible.
\section{Conclusion}

Several of the methods and tools presented in \emph{Project Management for
Information Systems}\cite{caye} were found to be useful in a project like ours,
while others were not.

Planning Poker has proven to be very useful due to its ease of use, needing
almost no introduction, as well as its proneness to start discussions about the
tasks being estimated, all without taking too much time to perform.

The use of a quality plan has reduced the time needed to perform quality checks
on program parts, freeing up more time to do actual work.

Using Scrum as a light-weight development framework in a small and volatile
project proved very useful, because it allows the project to evolve and change
over time, thus also allowing the requirements of the product to change. Its
agile nature makes it work well with several different estimation methods, and
allows for tasks to be estimated anew, multiple times, limiting the impact of
incorrect estimates.

\emph{risk management}
\emph{Planning}

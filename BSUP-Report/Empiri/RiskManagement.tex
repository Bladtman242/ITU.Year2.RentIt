\section{Risk Management}

In order to be prepared for what may go wrong during our project, we use risk management techniques.
We identified several risks, and placed them into several categories inspired by those used in Cadle
\& Yeates (should be reference). We have no suppliers to take into account, and the project is not
based on an actual business case (as it is a thought-up example to be used to test project management
methods).

Each risk has an impact, which is how much this issue could prolong the time spent in development,
if it was realized. We are using Cadle \& Yeates' proposed impact-size definitions, based on percentages:
\textbf{large impact} (more than 10\%), \textbf{moderate impact} (5-10\%), and \textbf{small impact}
(less than 5\%).

Additionally, the likelihood of the risk manifesting should also be taken into account. Here we use the
categories \textbf{high probability} (greater than 30\%), \textbf{medium probability} (10-30\%), and
\textbf{low probability} (less than 10\%).

Below is an excerpt of the most important risks (refer to the appendix for the full list). The risks are
identified with a string representing their category and a number to make the identification unique. The
full list of risks can be seen in Appendix \ref{app:risks-appendix}.

\begin{description}
    \item[T4 Development environment differs from Live environment] The environment present on
        developers' computers as well as the way WCF applications are tested locally, differs vastly
        from the way the code is deployed. This requires the maintenance of two different (but similar)
        configuration files. It is likely that there will, at some point during the project, arise a
        situation where there is some inconsistency between the two, resulting in improper tests. This
        could be everything from very easy to very hard to fix, depending on the situation.\newline
        \textbf{\emph{Impact: moderate; Likelihood: high}}
    \item[P2 Time constraints may be exceeded due to project overhead] Starting a project, especially when
        it is a small one, requires a relatively large amount of overhead (bureaucracy, preparation). This
        often requires for the entire team to assemble and agree on the terms by which they work. \newline
        \textbf{\emph{Impact: moderate; Likelihood: high}}
    \item[Req1 Requirements are changeable] As the requirements are a product of two teams (separated by
        space, and often time) discussing their wants and needs, it is very likely that they will be very
        volatile. Changing requirements early on is not a problem, but the later they change the more of an
        impact they will have on the development, as more of the product will have to be scrapped and redone.
        \newline
        \textbf{\emph{Impact: moderate; Likelihood: high}}
\end{description}

\begin{table}[t]
    \begin{tabular}{ | l | c | c | c | }
        \hline
        \backslashbox{Likelihood}{Impact} & \makebox[4em]{Small} & \makebox[4em]{Moderate} & \makebox[4em]{Large} \\[1em]
        \hline
        High & & \pbox{4em}{T4 P2 Req1} & \\[1em]
        \hline
        Medium & \pbox{4em}{Req2} & \pbox{4em}{T2} & \\[1em]
        \hline
        Low & \pbox{4em}{T3 P1} & \pbox{4em}{T1} & \\[1em]
        \hline
    \end{tabular}
    \caption{Table showing the spread of risks in the project. There are no risks with large impact, which
        leaves us free to concentrate on lesser risks.}
\end{table}

\subsection{Risk actions}
There are four types of actions possible when dealing with risks (according to Cadle \& Yeates):

\begin{itemize}
    \item Acceptance: Accepting the risk and doing nothing about it.
    \item Avoidance: Attempting to lower the likelihood of the risk.
    \item Mitigation: Attempting to lower the impact of the risk.
    \item Transfer: Attempt to move any potential damage caused by the risk to someone else.
\end{itemize}

We focused on mitigating the risks with the highest \verb+impact*likelihood+.

For \textbf{Risk T4}, the difference between development and live environments, there was not much we could
do. As we would not have access to the server before starting the project we had to accept this risk.

In order to avoid overhead in the project (\textbf{Risk P2}) we focused on choosing a development
methodology that would require little set-up, allowing us to start developing quickly. This is discussed in
more detail in Section \ref{sec:organizational}.

To mitigate changing requirements (due to several involved parties; \textbf{Risk Req1}) we focused on
thoroughly discussing requirements early on, in order to take as much as possible into account. We did not
expect this to bar any future changes, but hoped that doing a little more work up front we would have to
backtrack and redo less in future iterations.

Similar actions were taken for a few other risks. A lot of risks were accepted (Risks T3, P1 and Rel1) as
they did not seem to pose enough of a threat to justify the work required for mitigation or avoidance.

\textbf{(Note til analysatorere: Skriv endelig om hvor mange ting vi undgik grundet denne fantastiske analyse!)}
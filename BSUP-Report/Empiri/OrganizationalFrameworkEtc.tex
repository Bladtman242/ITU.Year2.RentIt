\section{Organizational framework and development approach}
\label{sec:EmpiriOrganizational}
\label{sec:organizational}

During the project we faced the challenge of potentially massive overhead involved in development.
This section attempts to explain how we handled this challenge, as well as which problems this way
of handling it caused.

As a team we had no say in what kind of organization we were situated in – it was pre-decided. Our
situation can be seen as a \emph{pure} project organization\cite{caye}, where individual project members are
pulled from several departments to focus on a project. In our case these departments are the
ITU and the SMU.

There are, however, other concerns to be had. None of the team members are working exclusively
on this project (due to other concurrent courses at their respective universities) and meeting
times can vary a great deal.

Using a software development methodology (used here as a catch-all term for approaches to
development, often consisting of several \emph{methods} or \emph{practices}\footnote{In other
words, using the definition found on Wikipedia: "A \textbf{software development methodology}
or \textbf{system development methodology} in software engineering is a framework that is used
to structure, plan, and control the process of developing an information system."
http://en.wikipedia.org/wiki/Software_development_methodology}) for a project is supposed to help to
improve quality of the product, as well as decrease the frequency of various issues. Different
methodologies have different focuses and, as a result of this, mitigate different kinds of issues.
Methodologies come in many shapes and sizes, and we considered several of these before deciding
on using Scrum internally in the Danish team.

Scrum uses an iterative approach (as opposed to a linear approach), where certain steps are
repeated rapidly and continuously to bring in more functionality step by step (instead of
introducing it all at once).

When using Scrum, less bureaucracy is required\cite{caye} which allows us to quickly set up a project
group and get started. As the project is rather small in scope, it seems ideal to avoid any overhead
that is not strictly required (this was also defined as a risk to the project, see Section \ref{sec:RiskManagement}).

Usually, a project would be split into 30-day sprints when using Scrum; however, due to the size of
our team and project, we chose to work with 7-day sprints instead. A sprint is a period in which the
team focuses on a pre-determined set of features. The set of features is explained as stories,
explaining how a user would interact with the application.

On meetings between each sprint, a \emph{backlog} for the entire project is maintained (more stories are
added, obsolete stories are discarded, stories with changing requirements are edited) and the stories
for each sprint are picked from this prioritized pool. The set of stories for a single sprint is known
as the \emph{sprint backlog}.

In a Scrum team there is a Scrum Master who ensures that all members of the teams follow the guidelines
of the methodology. This includes standing up at daily meetings (which we chose to have on a weekly basis),
maintaining the various backlogs, and more.

In our case, the Scrum Master had to take on multiple roles and also work as a developer, as we are a relatively
small team that would not have been effective otherwise (losing 20\% of our workforce).

While using scrum, we faced several issues.

Because of the volatile schedules of each team member, it was often hard to find days to meet and work
together. We ended up having just part of a single day every week, often meeting up in pairs or working
from home.

There were several factors resulting in some sprints having very active development and others being almost
completely stagnant (no work done). This had a lot to do with schedules of other courses colliding in
time-consuming ways.

We ended up spending a lot of time apart, working from home. It was hard to maintain a Scrum Master role from
afar because it is a role that requires observing the team. Keeping in touch through online communication did
make this possible, although it was far from a perfect solution.

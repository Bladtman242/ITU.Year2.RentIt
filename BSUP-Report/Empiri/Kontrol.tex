\section{Project Control}
\label{sec:EmpiriControl}
This section attempts to explain how we secured the quality of our project, as well as how we kept control of tasks, making sure that everything we set out to do was actually done.

\subsection{Quality Plan}
\label{sec:EmpiriQualityControl}
To ensure that our project kept a certain standard, we made a simple quality plan\cite[ch.~14.4.2]{caye} to describe what is considered good quality and how we should test the quality.

In order for a piece of the program to be accepted as having high enough quality, we
agreed that it must:

\begin{itemize}
    \item use responsible error handling;
    \item be well documented and readable by peers of the author;
    \item be tested with all inputs (equivalence groups); and
    \item fit in the API described in the design.
\end{itemize}

To check that the program lives up to the quality requirements, all tasks must go through a peer review before being accepted as done. The program will then have to be tested systematically to assure it upholds the quality requirements. If the program passes these tests, it is considered done and can be delivered.

The responsibility of the quality of a delivered piece of program lies with the author
and the reviewer. The reviewer should also review the tests made for the program and
the results of these tests. If the delivered program does not live up to the quality
requirements, both will be held responsible.

\subsection{Monitoring Progress}
\label{sec:EmpiriProgress}

Monitoring the progress of a project helps to determine if the deadlines can be upheld.
If this is not the case, it is important that this is discovered as soon as possible,
allowing new and more precise estimates to be made, and more realistic deadlines.

Our progress monitoring was based on two vital artifacts: the Scrum backlog and the network
diagram with critical path. The backlog is a very strong tool for monitoring the progress
of projects as it allows a project manager to easily get an overview of the current state
of the project.

With the sprint backlogs, the manager can see the progress of the individual tasks to zero 
in on what task exactly is lacking behind. Based on the backlogs we were also able to create 
burndown charts (can be seen in appendix \ref{app:burn}). These helped increase the ease 
with which the manager could get an overview of the projects and the individual sprints' progress. 
The burndown charts were available to all group members which allowed everyone to follow the progress.

The network diagram with its critical path is a good supplement to the backlogs because it
makes it easy to tell which activities should be prioritised if the manager needs to save
some time. Combined, these two artifacts function like the conventional time sheet as
described by Cadle \& Yeates\cite[ch.~11]{caye}.

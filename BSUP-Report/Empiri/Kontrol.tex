\section{Control}
\subsection{Quality Plan}
To ensure that our project kept a certain standard, we made a simple quality plan to describe what is considered good quality and how we should test the quality.

\textbf{Quality Requirements}\\
In order for a piece of the program to be accepted as having high enough quality, it must uphold the following:
\begin{itemize}
\item It must use responsible error handling
\item It must well documented and readable by peers of the author
\item It must be tested with all inputs (equivalence groups)
\item It must fit in the API described in the design
\end{itemize}


\textbf{Quality Checking}\\
To test that the program lives up to the quality requirements, all tasks must go through a peer review before being accepted as done. The program will then have to be tested systematically to assure it upholds the quality requirements. If the program passes these tests, it is considered done and can be delivered.

\textbf{Quality Responsibility}\\
The responsibility of the quality of a delivered piece of program lies with the author and the reviewer. The reviewer should also review the tests made for the program and the results of these tests. If the delivered program does not live up to the quality requirements, both will be held responsible and may be disciplined accordingly.

\subsection{Monitoring Progress}
During a project, monitoring the progress of the work being done is very important as this is the best way to tell if the deadlines can be upheld. If this is not the case, it is important that this is discovered as soon as possible, thereby allowing new and more precise estimates to be made, allowing for more realistic deadlines.

Our progress monitoring was based on two vital artifacts: the SCRUM backlog and the network diagram with critical path. The backlog is a very strong tool for monitoring the progress of projects as it allows a project manager to easily get an overview of the current state of the project. With the sprint backlogs, the manager can see the progress of the individual tasks to zero in on what task exactly is lacking behind. Based on the backlogs we were also able to create burndown charts. These helped increase the ease with which the manager could get an overview of the projects and the individual sprints' progress. The network diagram with its critical path is a good supplement to the backlogs because it makes it easy to tell which activities should be prioritised if the manager needs to save some time. Combined, these two artifacts function like the conventional time sheet as described by Cadle and Yeates.

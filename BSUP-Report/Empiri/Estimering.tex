\section{Estimation}
\label{sec:EmpiriEstimation}

We used a number of estimation methods introduced in the book \cite[ch.~9]{caye} and throughout the
course. This section attempts to explain these as well as the rationale behind our choices.

Once we had identified the activities in our project, with the work breakdown structure method,
estimates as to how long time the activities would take to complete could be made. In our project
and team we faced several challenges with regards to estimating accurately.

\begin{itemize}
\item The five of us had never worked together on a project before, which made it hard to factor
    in how time consuming communcation and teamwork in general would be. We had no previous experience
    or metrics to base our estimates on.
    
\item Part of the project was done by Singaporean students who we had not even met when we did the
    initial estimates. Not knowing how much experience they had in similar projects and with the web
    service technologies was a major complication in estimating.
    
\item We were working with technologies that were new to us. This made it hard to estimate activities
    involving these new technologies as we did not yet know how complicated they would be.
\end{itemize}

\subsection{Direct estimation}

As the work breakdown structure was already done, the natural choice for us was to use the \emph{direct
estimation method}\cite{caye}. The method bases its estimation on the activities discovered by the work
breakdown structure or project breakdown structure method. Each activity is  estimated separately, by
one or more qualified estimators. Eventually all of the activity estimates add up to the total project
estimate. 

For the estimation of the individual tasks we used a method called \emph{planning poker}, which is commonly
used in Scrum teams. In the method each team member has five pieces of cardboard with \textbf{small},
\textbf{medium}, \textbf{big}, \textbf{too small}, or \textbf{too big} written on them. Each task is
introduced and every member selects the piece of cardboard they believe to best describe the size of the task.
The people with the highest and the lowest estimate then explain their views on the task. After that the
process is repeated until all group members agree on an estimate. The final estimates were the basis for our
dependency diagram (Appendix \ref{app:dependencydiagram}).

We chose this method to ensure that everyone in the group got their say and because it seemed to work well in
the Scrum methodology as having estimates for all tasks is key in sprint planning (see Section
\ref{sec:EmpiriOrganizational} for an explanation of sprints as well as our general use of Scrum). To further
facilitate easy sprint planning we decided to estimate testing of tasks separately from the task itself, this
should also lead to better estimates in the future as it will create better metrics to reference.
\section{Estimation}
\label{sec:EmpiriEstimation}
Once we had indentified the activities in our project, with the work breakdown structure method, we wanted to make some estimates as to how long time the activities would take to complete. For several reasons the estimation was difficult to get accurate.
\begin{itemize}
\item First off, the five of us had never worked together on a project before, which made it hard to factor in how time consuming communcation and teamwork in general would be, in other words we had no previous experience or metrics to base our estimates on. 

\item Secondly, part of the project had to be done by Singaporean students who we had not even met when we did the initial estimates. Not knowing how much experience they had in similar projects and with the web service technologies proved a major complication in estimating.

\item  On top of these complications we were working with technologies that were new to us. This made it hard to estimate on activities involving these new technologies as we did not know how complicated they were.

\end{itemize}
\subsection{Direct estimation}
As the work breakdown structure was already done, the natural choice for us was to use the direct estimation method\cite{caye}. The method bases its estimation on the activities discovered by the work breakdown structure or project breakdown structure method. Each activity is  estimated separately, by one or more qualified estimators. Eventually all of the activity estimates add up to the total project estimate. 

For the estimation of the individual tasks we used a method called planning poker, which we were introduced to in a SCRUM introduction lecture. In the method each team member has five pieces of cardboard with "small", "medium", "big", "too small", or "too big" written on them. Each task is introduced and every member selects the piece of cardboard they believe to best describe the size of the task. The people with the highest and the lowest estimate then explain their views on the task. After that the process is repeated until all group members agree on an estimate. \\
We chose this method to ensure that everyone in the group got their say and because it seemed to work well in the SCRUM methodology as having estimates for all tasks is key in sprint planning. To further facilitate easy sprint planning we decided to estimate testing of tasks separately from the task itself, this should also lead to better estimates in the future as it will create better metrics to reference.


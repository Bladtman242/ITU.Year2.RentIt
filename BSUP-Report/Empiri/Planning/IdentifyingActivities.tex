In order to see the big picture, we decided to use an approach from Cadle\&Yeates\cite{caye} known as a \emph{work breakdown structure}. The work breakdown structure is organized in a hierarchical tree structure, where each new level in the tree is created by dividing a task into subtasks, as shown in figure \ref{fig:breakdown}.

\begin{figure}[hbtp]
	\includegraphics[scale=0.5]{./Empiri/Planning/img/wbslevels.png}
	\caption{Example work breakdown structure} \label{fig:breakdown}
\end{figure}

The division of tasks into smaller components is an iterative process, which stops when the resulting tasks are small enough to be considered a suitable assignment for one man or, more subjectively, when it simply does not make sense to divide it any further.

The use of a work breakdown structure required several discussions regarding the identification
of the work. Our situation left us especially prone to discussion as the problem domain was rather
new for us, making it difficult to conclude on what needed to be done.

Our work breakdown structure can be found in appendix \textbf{insert reference here}.
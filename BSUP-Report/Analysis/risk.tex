\section{Analysis of Risk Management}

There are two areas of concern when analyzing the effectiveness of risk management, and the first is
based more on qualified guesses than the other.

The first area is what issues or difficulties we avoided by doing a risk analysis early on, and what
issues we did not predict to have, but still ended up facing. Looking at these will give us an idea
of how much we gain from using a risk analysis.

The second area is issues when using the analysis itself, and how much effort it requires to complete
such an analysis.

These two areas together give us an idea about whether the \emph{benefits} of the risk analysis are
ultimately worth the \emph{effort} required.

\subsection{Effectiveness of Preventive Analysis}

In the empirical section we highlighted three different risks to be focused on and handled. These risks
will be the focus for determining the effectiveness of our risk analysis.

We place each of the risks considered in one of the categories \textbf{very effective}, \textbf{some
effect}, and \textbf{no effect}, to facilitate further discussion.

\subsubsection{P2: Project overhead}

Because we realized early on that project overhead was a real issue, we planned around this and succesfully
avoided this risk entirely. Our team worked in a highly agile fashion, and the choice of methodology helped
us stay on target.

Avoiding this risk proved \textbf{very effective}.

\subsubsection{Req1: Requirements are changeable}

Discussing requirements and common interfaces requires involvement of several parties. We had several
discussions with the Singaporean team, that led to a firm belief that our contact surfaces were going
to stay more or less the same.

As it turns out, either due to bad communication or misunderstanding of expectations, the Singaporeans
did not see these early discussions as anything close to final.

During the final week of collaboration a lot of requests for changes to the common interfaces surfaced,
putting a heavy load on the Danish team and requiring some minor changes to the internal data structure.

The mitigation of the risk was not entirely successful, and this shows that however much is done to mitigate
or avoid a risk, it is impossible to be entirely sure.

The mitigation did, however, make sure the most central interfaces (and hence most of the internal datastructure)
could be preserved. The initial discussions did have \textbf{some effect}, however not as much as desired.

\subsubsection{T4: Development differs from Live}

The final risk we chose to focus on had to be accepted, because we were unable to do
anything to handle it before starting the project. Having done the analysis, however, was a great help
as it made us aware of this potential threat to development stability.

Because of this awareness we focused on getting the live and development environments as synchronized as
possible as well as developing a clear protocol for updating configuration files for the servers, in
order to avoid any discrepancy.

Had we not had these protocols in mind it is very likely that this could have developed into a serious
roadblock for development. Although any such risk is entirely hypothetical (as we have not tried this
situation without first doing a risk analysis) it seems fair to assume that it was a nice safe-guard to
have in place.

Spending time on this risk had \textbf{some effect}.

\subsubsection{Unavoided risks}

Because the risks were prioritized and only the first few handled, some risks could come up as issues and
hit development with full force. One such case was a risk that we had actually anticipated, but chosen
not to handle: the fact that we were developing a new type of application (\textbf{Risk T2}).

Unknown technology may be more complex than first anticipated, and this proved to be true for our situation.
We had only anticipated a few possible issues, but had not anticipated the one that hit us.

As it turns out it was difficult to send out responses with the proper HTTP error codes when using a IIS based
WCF service. As this wasn't exactly blocking further development, the impact was limited, but it was an issue
that took a substantial amount of time to fix.

This shows that risk management can never eliminate all issues.

\subsection{Effort Required}

Our risk analysis was performed as a group discussion, where any and all concerns were brought up to scrutiny.
As it turns out, with a project of the size we undertook, the number of concerns was limited. After an initial
discussion we had a couple of days to come up with any additional risks or revisions and then a final discussion
to implement these.

All in all the effort required was rather small, and the planning was done in a matter of hours. Efforts to actually
mitigate or avoid risks were more substantial, but could mostly be seen as development time. This time was kept
negligible by choosing to only focus on a few risks. This could have been adjusted by trying to avoid or mitigate
additional risks.

\subsection{Cost vs Benefit}

With the amount of effort we put into the analysis (cost) and the effect it yielded (benefit), there can be no doubt
that this practice of analyzing risks is very recommendable, with some to high effectivity and little effort.

In a situation with as few potential risks as ours it would be recommendable to attempt avoiding or mitigating more
risks than we did in our particular situation. We were hit by a risk that was placed in the second-highest category,
moderate impact, medium likelihood. If we had chosen to handle the risk in this category a great challenge may have
had less of an impact.
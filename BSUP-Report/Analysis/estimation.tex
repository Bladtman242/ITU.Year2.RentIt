\section{Analysis of Estimation Methods}
\label{sec:AnalysisEstimation}
This section explains how planning poker worked as an estimation method in this
particular project while also considering protential alternatives.

Planning poker allows every member of the group to give their initial estimate
of a task uninfluenced by the estimates given by the rest of the group, because
everyone reveals their estimate at the same time. This gives everyone the chance
to explain their estimate, perhaps making other group members realize that their
initial estimate was wrong, which was the case during several of our Scrum
meetings.

It also invites discussions of the different tasks to be estimated which can
help to clarify what the task in question is about (people may reach different
estimates because they have understood the tasks differently), which is a
definite quality we found in using this estimation method.

While time consuming, we feel that using planning poker has definitely paid off,
both in terms of estimating and in terms of understanding and defining our
tasks.

\subsection{Alternative Estimation Methods}
Seeing how well the planning poker method worked, the Delphi
technique\cite{caye} might also have been useful as it follows the same idea of
letting people change their estimates depending on estimates given by the rest
of the group. However, with the Delphi technique being anonymous in the sense
that each estimator gives their estimate anonymously, estimators will not be
given the chance to explain why they have estimated a given task they way they
did. Being anonymous, though, it allows estimators to be less concerned about
changing their estimate to comply with the rest of the group if they are certain
their estimate is the right one. In some ways this method might be better than
the planning poker method because of its anonymity, but being anonymous might
also make it less ideal in some circumstances (for example, if two estimators
disagree and both refuse to change their estimates).

Another estimation method that might be suitable for our type of project is the
Analysis Effort method\cite{caye} in which the team estimates the effort needed
for the analysis part of each individual program function. Then, by considering
a series of factors such as team size, familiarity with the type of work, and
complexity, the ratio between analysis, which has now been estimated, and
design, coding, and testing. In total, this gives an estimates of the entire
project and subtasks of which it is composed.

This is a somewhat stricter estimation method than planning poker and the Delphi
technique as the factors are less up to interpretation than simply giving a
number to represent an estimate. For example, if the team size is known, this
factor will always be the same. Leaving less to imagination and interpretation
of the factors used to determine the estimate could be useful in groups that are
less experienced in estimates (like our own) as there is a template to follow.

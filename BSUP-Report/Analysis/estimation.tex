\subsection{Analysis of Estimation Methods}
\subsubsection{How did it go?}
Planning poker allows every member of the group to give their initial estimate of a task uninfluenced by the estimates given by the rest of the group, because everyone reveals their estimate at the same time. This gives everyone the chance to explain their estimate, perhaps making other group members realize that their initial estimate was wrong. This was the case during several of our SCRUM meetings.

\subsubsection{Alternative Estimation Methods}
Seeing how well the planning poker method worked, the Delphi technique might also have been useful as it follows the same idea of letting people change their estimates depending on estimates given by the rest of the group. However, with the Delphi technique being anonymous in the sense that each estimator gives their estimate anonymously, estimators will not be given the chance to explain why they have estimated a given task they way they did. Being anonymous, though, allows estimators to be less concerned about changing their estimate to comply with the rest of the group, if they are certain their estimate is the right one. In some ways this method might be better than the planning poker method because of its anonymity, but being anonymous might also make it less ideal in some circumstances (for example, if two estimators disagree and both refuse to change their estimates).


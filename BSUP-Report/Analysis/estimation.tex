\section{Analysis of Estimation Methods}
\label{sec:AnalysisEstimation}
This section explains how planning poker worked as an estimation method in this
particular project while also considering potential alternatives.

Planning poker allows every member of the group to give their initial estimate
of a task uninfluenced by the estimates given by the rest of the group, because
everyone reveals their estimate at the same time. This gives everyone the chance
to explain their estimate, perhaps making other group members realize that their
initial estimate was wrong, which was the case during several of our weekly Scrum
meetings.

It also invites discussions of the different tasks to be estimated which can
help to clarify what the task in question is about (people may reach different
estimates because they have understood the tasks differently), which is a
definite quality we found in using this estimation method.

The process of estimating through this method could potentially end up requiring
a lot of time as all members of the team must agree. We found that our team size
of 5 (or any team with 5 estimators) was very fitting, as the dicussions only
occasionally dragged on.

Despite being time consuming, we feel that using planning poker has definitely
paid off, both in terms of estimating and in terms of understanding and defining
our tasks.

\subsection{Alternative Estimation Methods}
Seeing how well the planning poker method worked, the Delphi
technique\cite[ch.~9.3.5]{caye} might also have been useful as it follows the same idea of
letting people change their estimates depending on estimates given by the rest
of the group. However, with the Delphi technique being anonymous in the sense
that each estimator gives their estimate anonymously, estimators will not be
given the chance to explain why they have estimated a given task the way they
did. Being anonymous, though, it allows estimators to be less concerned about
changing their estimate to comply with the rest of the group if they are certain
their estimate is the right one.

In some ways this method might be better than the planning poker method because
of less risk of feeling peer-pressured, but being anonymous might also make it
less ideal in some circumstances (for example, if two estimators disagree and
both refuse to change their estimates).

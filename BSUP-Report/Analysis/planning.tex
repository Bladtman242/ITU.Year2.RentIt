\section{Analysis of our Planning Toolbox}
This section explains our thoughts on the different tools used in planning our project, as well as how useful they were.
\label{sec:AnalysisPlanning}
\subsection{Work to do}
In an attempt to discover, and organize, the work in front of us, a work
breakdown structure was created. It created a rhetorical turmoil, as the
diagram itself, rather than the structure it abstracts, became subject of our
focus and conversation. At the time, we were not yet conscious of what work
would and wouldn't need to be done, and so the diagram became a little shallow
and vague. In retrospect, a product breakdown structure may have been the
appropriate tool at the time, as it would have allowed us to focus on
\emph{what} needed to be produced, and delay making decisions on \emph{how} to
produce it\cite[ch.~8.3,~8.4]{caye}.

\subsection{The Path to Success}
Something that \emph{did} work out for us is the identification and satisfaction
of dependencies. Our initial time estimates may have been off, but that hardly
lessens the advantage of knowing the order in which to approach activities.
While seemingly trivial, failure to assert activity dependencies can cause much
frustration and wasted time. Having previously learned this the hard way, there
was little inclination to discuss the validity of dependency analysis.

While network diagramming is a fairly rich and explicit representation of
activities and inter-activity dependencies, they are not easy to read.  Gantt
diagrams on the other hand, are easy to read, perhaps because they are much less
explicit. Concretely, Gantt charts say nothing about the dependencies - but
express only the order - of activities. As such, Gantt charts may not be very
helpful in the planning process, but are useful for describing the plan itself.
\cite[ch.~8.6]{caye}

While not bullet proof, this kind of planning can decide success or
demise\cite[ch.~8]{caye}, especially for larger projects, where the risk of
losing the overview is increased. The bigger picture painted by the use of
these tools can uncover hidden issues, and help prepare for more obvious ones.

\section{Analysis of our Planning Toolbox}
\subsection{Work to do}
In an attempt to discover, and organize, the work in front of us, a work
breakdown structure was created. It created a rhetorical turmoil, as the diagram
itself, rather than the structure it abstracts, became subject of our focus and
conversation. At the time, we were not yet conscious of what work would
and wouldn't need doing, and so the diagram became a little shallow and vague.
In retrospect, a product breakdown structure may have been the appropriate tool
at the time, allowing us to focus on \emph{what} needed to be produced, and delay
making decisions on \emph{how} to produce it\cite{caye}.

\subsection{The Path to Success}
Something that \emph{did} work out for us is the identification and satisfaction of
dependencies. Our initial time estimates may have been off, but that hardly
lessens the advantage of knowing the order in which to approach activities.
While seemingly trivial, failure to assert activity dependencies can cause much
frustration and wasted time. Having previously learned this the hard way, there
was little inclination to discuss the validity of dependency analysis.

\emph{something about gantt charts. Yet to be}

While no where near bullet proof, this kind of planning can decide success or
demise\cite[chpt.~8]{caye}, especially for larger projects, where the risk of
losing the overview is increased. The bigger picture painted by the use of
these tools can uncover hidden issues, and help prepare for more obvious ones.

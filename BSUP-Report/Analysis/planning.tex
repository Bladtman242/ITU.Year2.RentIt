\section{Analysis of the Planning Process}
\label{sec:AnalysisPlanning}

This section explains our thoughts on the different tools used in planning our
project, as well as how useful they were.

\subsection{Work breakdown structure} In an attempt to discover and organize
the work in front of us, a work breakdown structure was created. It created a
rhetorical turmoil as the diagram itself, rather than the structure it
abstracts, became subject of our focus and conversation.

At the time, we were not yet entirely clear on what work would need to be done,
and so the diagram was shallow and vague. Our situation left us especially
prone to discussion as the problem domain was rather new for us, making it
difficult to conclude on what needed to be done. In retrospect, a product
breakdown structure may have been the appropriate tool at the time. Making a
tree of products rather than activities would have allowed us to focus on
\emph{what} needed to be produced, and delay making decisions on \emph{how} to
produce it\cite[ch.~8.3,~8.4]{caye}.

\subsection{Activity dependencies}
Something that \emph{did} work out for us was the identification and satisfaction
of dependencies. Our initial time estimates were imprecise, but we still got the
advantage of knowing the order in which to approach activities. While seemingly
trivial, failure to assert activity dependencies could have caused much
frustration and wasted a lot of time. Having previously learned this the hard way,
there was little inclination to discuss the validity of dependency analysis.

While network diagramming is a fairly rich and explicit representation of
inter-activity dependencies, they are not easy to read. Gantt
diagrams, on the other hand, are easy to read, perhaps because they are more
abstract. Gantt charts do not describe things such as slack and the critical
path, but provide a less information-cluttered overview. As such, Gantt charts
may not be very helpful in the planning process, but are useful for describing
and quickly getting an idea of the plan itself.

Gantt diagrams can show much more than just activities over time, and can quickly
get cluttered\cite[ch.~8.6]{caye}. We chose to show dependencies, but other than
that stuck to the most basic formula.

While not bulletproof, this kind of planning can decide success or
demise\cite[ch.~8]{caye}, especially for larger projects, where the risk of
losing the overview is increased. The bigger picture painted by the use of
these tools can uncover hidden issues, and help prepare for more obvious ones.

\section{Analysis of Development Process}
This section provides an analysis of the one development methodology used during our projects, namely Scrum. It mainly focuses on explaining our thoughts on using Scrum and on tailoring it to our needs.
\label{sec:AnalysisOrganizational}

Scrum is a widely used and very popular software development methodology. This shows in the amount
of articles found about it online and discussions about how to use (or even improve) it. This means
that there are many different versions of Scrum based on different company and team profiles. They
all inherit the same structure, but some details vary.

One thing that has changed over time for Scrum is the length of sprints. The traditional recommendation
for 30-day sprints (as in Cadle\&Yeates\cite[ch.~6.3.3]{caye}) have been replaced by a guideline that says that
each iteration should be "no more than four weeks each (the most common is two weeks)"\cite{scrumprime}.

Seeing as we are a very small team working on a relatively small project, it seems natural to have as 
short iterations as possible: The shorter the iteration, the more \emph{agile} we get - we are quicker
to adapt to changing situations. With a total project length of less than three months it would be
dangerous to lock in sprint backlogs 30 days at a time.

We opted for more iterations of a shorter duration and landed on a week as a well-fitting size. This
proved a great help as we had many changes to requirements, sometimes several in the same week. Because
we could rapidly adjust our course this had little impact on our work.

\subsection{Stand Up Meetings}

In our weekly meetings, where we would follow up on what had been done as well as discuss what would be
next, we did not always actually \emph{stand up}. The meetings were fairly rare, but a great way of
motivating the team. If it had in any way been possible to have these meetings more than weekly this would
likely have added to our productivity. Unfortunately this was not possible due to our schedules.

At the meetings where we were standing we noticed an increase in pace, focus and motivation: all team members
were more eager to get started with the day's work, and do it well. A heightened focus on standing would
likely have had a positive effect.

\subsection{Stagnant sprints}

Sprints in which no work was done were a real problem in the project. We did not fall behind schedule because
of these, but it did make development highly irregular, resulting in more meetings where all team members
would be brought back up to speed.

Additionally when only working in spurts, these spurts were much more intense. Some sprints were stagnant, but
others required long work days, resulting in a decline in our productivity\footnote{This is a well-known issue
that is covered in Kent Beck's \emph{Extreme Programming Explained}\cite{xpe} as well as in other fields of
research\cite{workingtime}.}.

To avoid stagnant sprints, we could have made a schedule over how many hours each member could work each week
and make sure that some work was done each week. This would have solved most of our problems in this area. It
would immediately reduce work in active sprints, and lessen time spent in meetings.

On the other hand most factors that added up to stagnation were unpredictable and out of our hands so the effect
of creating such a schedule is dubious. Having a schedule is one thing, sticking to it is another. With several
concurrent projects for each team member it is hard to have a regular schedule.

\subsection{Final words}

All in all, using Scrum helped us avoid a serious risk for the development timeframe (see Section 
\ref{sec:RiskManagement}) as well as provide a general framework.

We made some helpful changes to the process, but did not enforce some important parts of the methodology, which
would have been helpful had they been enforced.
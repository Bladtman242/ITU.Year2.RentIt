\section{Considerations wen applying shit}
In this section, we will go through a couple of considerations we wish we had
been more aware of during the project.

We have found that in small projects like this, it is important to be able to
see the project progress at all times. The Scrum backlogs are an intuitive way
of monitoring the progress of a project and that is something we would
recommend using in situations similar to ours. Additionally we recommend having
a quality plan to make it easy to assess when a task is done.

We also learned that it is hard to make good estimates, and it is a fine
balance between time spent estimating and the needed accuracy of the estimates.
The use of a work breakdown structure followed by planning poker worked well
for us, but we suspect this to be quite time consuming in larger projects as
the task grows exponentially with the size of the project, and therefore we
would only recommend it in smaller projects. After having estimated, we
recommend to make a plan for the project, because without this you cannot
easily recognize if you are off schedule.

Scrum seems like a solid organizational framework which works well in projects
like ours. We recommend it for smaller projects in which an iterative approach
can be used and it is definitely an approach we will use in other similar
projects.

Additionally, we recommend being careful when doing risk management in a
project with limited man hours as it is easy to spend more time analysing risks
than what can be saved by handling these. However, an appropriate amount of
risk management can prove invaluable, and should be considered for all
projects..

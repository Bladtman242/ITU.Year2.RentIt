\section{Considerations when applying shit}
In this section, we will go through a couple of considerations we wish we had
been more aware of during the project.

We have found that in small projects like this, it is important to be able to
see the project progress at all times. The Scrum backlogs are an intuitive way
of monitoring the progress of a project and that is something we would
recommend using in situations similar to ours. Additionally we recommend having
a quality plan to make it easy to assess when a task is done.

We also learned that it is hard to make good estimates, and it is a fine balance
between time spent estimating and the needed accuracy of the estimates. 
Imprecise estimates are to be expected, and they will become better as the team
gets more familiar with each other and the project. The use of a work breakdown
structure followed by planning poker worked well for us, but we suspect this to
be quite time consuming in larger projects as the task grows exponentially with
the size of the project, and therefore we would only recommend it in smaller
projects. After having estimated, we recommend to make a plan for the project
(we used both a dependency diagram and a Gantt chart), because without this you
cannot easily recognize if you are off schedule.

Scrum seems like a solid organizational framework which works well in projects
like ours. We recommend it for smaller projects in which an iterative approach
can be used and it is definitely an approach we will use in other similar
projects.

Additionally, we recommend being careful when doing risk management. In a
project with limited man hours as it is easy to spend more time analysing risks
than what can be saved by handling these. On the other hand, there are risks we
deemed unnecessary to accommodate, that we wish we had. Finding the right
balance is hard, and might be worth investigating methods for faster risk
management.

\section{Reflection}

We have found that in small projects like this, one of the most important
things is to be able to see the project progress at all times. The Scrum
backlogs are a really intuitive way of monitoring the progress of a project and
it works really well with small projects, so that is definitely something we
would recommend using in situations similar to ours.

To make the backlogs useful it is very important that the estimates produced
are accurate. We learned that this is a hard thing to do well and it is a fine
balance between time spent estimating and the needed accuracy of the estimates.
The use of a work breakdown structure followed by planning poker worked well for
us, but it is quite time consuming in larger projects as the task grows exponentially with the
size of the project, and therefore we would only recommend it in smaller
projects.

In general we were really happy with working with Scrum, it seems like a really
solid organizational framework which works very well in projects like this. We
recommend it for all smaller projects in which an iterative approach can be
used. 

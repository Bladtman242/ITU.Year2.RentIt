\section{Introduction}

This report is written by a team of five students at the IT-University of Copenhagen (ITU)
studying Software Development on second year (4th semester).

The authors of this report are Niclas Benjamin Tollstorff (nben@itu.dk), Niels Liljedal
Christensen (nlch@itu.dk), Sigurt Bladt Dinesen (sidi@itu.dk), Kristian Brink Gansted
(kbri@itu.dk), and Niels Roesen Abildgaard (nroe@itu.dk).

The project used as a basis for this report, was done in collaboration with a group of
students from the Singapore Management University (SMU), and had the goal to develop an
application that would allow users to buy and rent films and music. The application consisted
of a server developed by the ITU students and a browser-based client developed by the SMU
students.

The Singaporean team was made up of Chai Ching Hsiang Robert, Tan Kah How Kelvin, and
Chong Wen Xiong Nick.

As a part of the course “System Development and Project Organisation”, we have learned
about several tools and methods to use in a software development context.

The basis for these methods is the book “Project Management for Information Systems”
(5th edition) by James Cadle and Donald Yeates, which explains these methods as well
as which situations they may be used in.

We will detail various problems faced during our project and discuss which of Cadle and
Yeates' methods can be used to overcome them. In the form of a case study, this report
aims to detail what problems may arise in a software development project, as well as our
experiences with using project organization and development methods to overcome these
issues. Issues that arose during the project are listed and paired with the methods we
used to overcome them.

Finally we will analyze the effectiveness of all the methods used with an aim to determine
the usefulness of each method. The analysis is based on the team’s experiences with the
methods and is purely qualitative.

We will, throughout the report, describe and discuss our experiences with:

\begin{itemize}
    \item project planning through \emph{work breakdown structure} and \emph{identification of
        dependencies}, resulting in a plan in the form of a Gantt diagram (Sections
        \ref{sec:EmpiriPlanning} \& \ref{sec:AnalysisPlanning});
    \item risk management through \emph{risk analysis}, as well as actual handling of risks
        (Sections \ref{sec:EmpiriRiskManagement} \& \ref{sec:AnalysisRiskManagement});
    \item estimation of features through \emph{planning poker} (Sections
        \ref{sec:EmpiriEstimation} \& \ref{sec:AnalysisEstimation});
    \item quality control, using a \emph{quality plan} (Sections \ref{sec:EmpiriQualityControl}
        \& \ref{sec:AnalysisQualityControl});
    \item progress monitoring, keeping in mind the initial planning and estimation (Sections
        \ref{sec:EmpiriProgress} \& \ref{sec:AnalysisProgress}); and
    \item choice of development framework (or methodology) for a small project (Sections
        \ref{sec:EmpiriOrganizational} \& \ref{sec:AnalysisOrganizational}).
\end{itemize}
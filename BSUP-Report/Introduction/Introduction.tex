\section{Introduction and Problem Statement}

This report is written by a team of five students at the IT-University of
Copenhagen (ITU) studying Software Development on second year (4th semester).

The project used as a basis for this report was done in collaboration with a
group of students from the Singapore Management University (SMU), and had the
goal to develop an application that would allow users to buy and rent films and
music. The application consisted of a server developed by the ITU students and
a browser-based client developed by the SMU students.

The Singaporean team was made up of Chai Ching Hsiang Robert, Tan Kah How
Kelvin, and Chong Wen Xiong Nick.

As a part of the course \emph{System Development and Project Organisation}, we
have learned about several tools and methods to use in a software development
context.

The basis for these methods is the book \emph{Project Management for
Information Systems} (5th edition) by James Cadle and Donald Yeates, which
explains these methods as well as which situations they may be used in.

We will detail project management methods used during our project, and problems
that may arise when using these. In the form of a case study, this report looks
into modifications to the methods used, and how these modifications help the
project. Issues that arose during the project are listed and paired with the
methods we used to overcome them.

Finally we will analyze the effectiveness of all the methods used with an aim
to determine the usefulness of each method. The analysis is based on the team’s
experiences with the methods and is purely qualitative.

Throughout the report, we will describe and discuss our experiences with:

\begin{itemize} \item risk management through \emph{risk analysis}, as well as
		actual handling of risks (Sections
	\ref{sec:EmpiriRiskManagement} \& \ref{sec:AnalysisRiskManagement});
	\item project planning through \emph{work breakdown structure} and
		\emph{identification of dependencies}, resulting in a plan in
	the form of a Gantt diagram (Sections \ref{sec:EmpiriPlanning} \&
	\ref{sec:AnalysisPlanning}); \item estimation of features through
		\emph{planning poker} (Sections \ref{sec:EmpiriEstimation} \&
		\ref{sec:AnalysisEstimation}); \item choice of development
		framework (or methodology) for a small project (Sections
	\ref{sec:EmpiriOrganizational} \& \ref{sec:AnalysisOrganizational});
	\item quality control, using a \emph{quality plan} (Sections
		\ref{sec:EmpiriQualityControl} \&
		\ref{sec:AnalysisQualityControl}); and \item progress
	monitoring, keeping in mind the initial planning and estimation
	(Sections \ref{sec:EmpiriProgress} \& \ref{sec:AnalysisProgress}).
\end{itemize}

This lead us to the following \textbf{problem statement}:
\begin{quote}
\emph{
How can Scrum and tools from the areas of risk management, project planning, estimation, quality control, and
progress monitoring be useful in a small software development project?
}
\end{quote}

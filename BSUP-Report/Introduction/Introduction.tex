\section{Introduction}

The report is written by a team of five students at the IT-University of Copenhagen (ITU) studying Software
Development, currently on our second year (4th semester).

As basis for this report, we will be working with a group of students from the Singapore
Management University (SMU) to develop an application that allows users to buy and rent
movies and songs. The application consists of a server developed by the ITU students
and a browser-based client developed by the SMU students.

The authors of this report are Niclas Benjamin Tollstorff (nben@itu.dk), Niels Liljedal
Christensen (nlch@itu.dk), Sigurt Bladt Dinesen (sidi@itu.dk, Kristian Brink Gansted
(kbri@itu.dk), and Niels Roesen Abildgaard (nroe@itu.dk).

The Singaporean team is made up of Chai Ching Hsiang Robert, Tan Kah How Kelvin, and
Chong Wen Xiong Nick.

As a part of the course “System Development and Project Organisation”, we have learned
about several tools and methods to use in a software development context.

The basis for these methods is the book “Project Management for Information Systems”
(5th edition) by James Cadle and Donald Yeates, which explains several of these as well
as which situations they may be used in.

This report details various problems we faced during our project and what
methods were used to overcome them. In a use case study, the report aims to
detail what problems arose during our project, and which methods are
recommended in such situations.  Issues that arose during the project are
listed and paired with the methods we used to overcome them.

The effectiveness of all the methods used is analyzed with an aim to determine
how useful each method is. The analysis is based on the team’s experiences with
the methods and is purely qualitative.

We also examine how features of organizational structures and methodologies
(such as Scrum) can be used to improve product quality and deadline-adherence
in a small software-development project.
\subsection{Risks}
\label{app:risks-appendix}

\subsubsection{Technical Risks}
\begin{description}
    \item[T1 New development environment] In the project we will be working with an unfamiliar
        framework (.NET’s WCF) and server technology (IIS). If any problems arise with these
        technologies they will considerably lengthen the project scope as no member of the team
        would have the required familiarity to quickly solve such a problem. On the other hand
        it is not very likely that any problems should arise as WCF is used through C\#, a
        programming language the entire team is familiar with, and the interface to the IIS-server
        is very much like using Windows on a desktop computer.\newline
        \textbf{\emph{Impact: moderate; Likelihood: low}}
    \item[T2 New type of application] None of the developers have previously developed a WCF-based
        RESTful service. This technology will require some self-education on various points. It is
        fairly likely that the team would end up in situations with lack of knowledge, and this may
        take a considerable amount of time to get around.\newline
        \textbf{\emph{Impact: moderate; Likelihood: medium}}
    \item[T3 New language] Few of the developers have ever worked with the JavaScript Object Notation,
        which will be a vital part of the service built. The syntax, however, is very simple and easy
        to learn, so it seems unlikely that this would amount to an actual problem.\newline
        \textbf{\emph{Impact: small; Likelihood: low}}
    \item[T4 Development environment differs from Live environment] The environment present on
        developers' computers as well as the way WCF applications are tested locally, differs vastly
        from the way the code is deployed. This requires the maintenance of two different (but similar)
        configuration files. It is likely that there will, at some point during the project, arise a
        situation where there is some inconsistency between the two, resulting in improper tests. This
        could be everything from very easy to very hard to fix, depending on the situation.\newline
        \textbf{\emph{Impact: moderate; Likelihood: high}}
\end{description}


Planning and resource risks
•	P1 Time constraints may be exceeded due to scope of project
The size of the project may have been underestimated, and the project may require more man-hours than originally planned. Due to the actual (small) scope of the project this is not likely, and if it did happen it is likely that the team members would be able to move around their schedule to put in more work, so the impact would not be high.
Impact: small; Likelihood: low
•	P2 Time constraints may be exceeded due to project overhead
Starting a project, especially when it is a small one, requires a relatively large amount of overhead (bureaucracy, preparation). This often requires for the entire team to assemble and agree on the terms by which they work.
Impact: moderate; Likelihood: high
Requirement risks
•	Req1 Requirements are changeable 
As the requirements are a product of two teams (separated by space, and often time) discussing their wants and needs, it is very likely that they will be very volatile. Changing requirements early on is not a problem, but the later they change the more of an impact they will have on the development, as more of the product will have to be scrapped and redone.
Impact: moderate; Likelihood: high
•	Req2 Requirements lack detail or are ambiguous
When discussing requirements the final version may be subject to some ambiguity, often the result of earlier misunderstandings. This may result in one team implementing a feature that will not work with what the other team has implemented. It is fairly likely that such a discrepancy will develop during global software development, but as the overall idea will most likely be preserved it is unlikely that fixing such an issue will amount to very much work.
Impact: small; Likelihood: medium

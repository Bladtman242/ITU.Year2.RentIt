\section{Problem Statement}
\label{sec:problemstatement}
double question marks (??) surrounds entries that are incomplete, or should be
subjected to scrutiny

Background for the project:
For the ITU course »Second Year Project: Software Development in Large Teams with
International Collaboration-F2013«
we are to design, produce, and perform the relevant domain-analysis for,
an online rental/purchase service in cooperation with a group from the Singapore
Management University.

The report:
This report attempts to discuss how the tools and theory taught in the
ITU course »System Development and Project Organisation« and in the
course book, »Project Management for Information Systems«, 5. edition, by 
James Cadle \& Donald Yeates can be used to improve the process of a software development project.

The discussion is based on the development of the aforementioned
rental-system, which is used as a case.
The main problems of the report are described by the following items:
How can the features of organization structures, ??such as a
programme and project support office??, be used to improve
product quality and deadline-adherence in a small,
??internationally distributed (this is actually called global
software development, or multi-site)??,  software development project.

How can SCRUM be used to improve the process of, and 
deadline-adherence in, a small, internationally distributed,
software development project.

What challenges may arise from team working, in particular
using SCRUM, in a distributed team?

Which project management and project analysis documents can
provide useful support for such a project, and which might
prove unnecessary.

\subsection{Requirements}
As part of the analysis, we need a set of requirements for our product to establish what users will and will not be able to do with the system. This section describes and discusses our choice of requirements and use cases.


There are two types of users: users and managers. A manager’s rights are a superset of a regular user’s rights. In addition, managers do not have to purchase anything.


Our use cases, which describe in more detail what users and managers must be able to do with the system, are shown in the appendix.

\subsubsection{Security and other Design Decisions}
When deciding on the requirements for the system, several criteria are considered, and some requirements which might be crucial in a real-life product are left out. These decisions are mainly made on the basis that this is an educational project, making these requirements more or less irrelevant to the project:
\begin{itemize}
\item We decided to leave out user authentication in our web services, so given the URL, anyone can download a movie our song without authenticating. The data structure does, however, support user authentication with each user having a password.
\item In a real-life renting service such as iTunes, once a user starts playing the movie that she has just rented, the movie will only be available for viewing in the next 48 hours. When a user rents and downloads a movie or song using our service, they can keep the file for as long as they want. However, the file is only available for download for a limited period of time.
\item Naturally, we have not implemented a real payment service but rather a "fake" balance where people can deposit money to their account, for use when purchasing or renting movies and songs.
\end{itemize}
\subsection{Requirements}
\label{sec:requirements}
As part of the analysis, we need a set of requirements for our product to
establish what users will and will not be able to do with the system. This
section describes and discusses our choice of requirements and use cases.

We did not immediately arrive at our current requirements. They have changed
over a series of iterations in which we revised and discussed them internally in
the group, as well as with the Singaporean team. For example, our first draft of
requirements did not include songs at all. It was not until the Singaporean team
told us that they were required by their professor to include songs in their
service that this became part of our requirements. The Singaporean team also
made the suggestion that media files should include ratings and view count to allow us to
sort by rating and popularity. In general, the requirements were almost
exclusively changed on request from the Singaporean team.

There are two types of primary actors: users and managers. A manager’s rights
are a superset of a regular user's. 

In the following requirements document, the word \emph{user} refers to an
existing, registered user. Artifact shall refer to a movie or song (the
products available for rent/purchase in the system). For the sake of this
description, the artifact properties are divided into two categories:
\emph{intrinsic} or \emph{extrinsic} to the artifact.

\emph{Intrinsic}
properties are those describing the artifact. Examples are title, URI to image,
and average rating. These properties are decided by the system actors, and are
not necessarily present. A manager may choose not to include an image URI, and
new artifacts may have no ratings.

\emph{Extrinsic} parameters are those
describing the artifacts from the system's perspective. Examples are id, and
view count. These parameters are completely determined by the system, and are
present regardless of actor interaction.

\textbf{Artifacts:}
\begin{enumerate}
\item A movie must have intrinsic properties describing its title, year of
	release, director(s), and genre(s).
\item A song must have intrinsic properties describing its title, year of
	release, artist(s), album title, and genre(s).
\item An artifact must have an extrinsic property describing its unique id.
\item An artifact must have an intrinsic property to describe an URI pointing to
	an image (E.g. cover image for movies).
\item An artifact must have a property to describe its average
	user-rating. This property is intrinsic (determined by actors, may be
	non-present) but only individual ratings are set by by actors, the
	average is maintained by the system, and it must not be an option for an
	actor to edit the average directly.
\item An artifact must have an intrinsic property for a textual description.
\item An artifact must have a property to describe both a purchase price, and a
	rental price. These are intrinsic (determined by the actors) but their
	presence must be enforced by the system. \label{itm:priceReq}
\item An artifact must have properties to describe its view count. I.e. number
	if times the artifact has been viewed or downloaded.
\newcounter{enumTemp}
\setcounter{enumTemp}{\theenumi}
\end{enumerate}

\textbf{Users:}
\begin{enumerate}
\setcounter{enumi}{\theenumTemp}
\item A (non) user must be able to register an account.
\item A user must be able to login to her account.
\item A user must be able to edit her account.
\item A user must be able to close her account.
\item A previous user must be able to reopen her closed account.
\item A user may deposit money into her account, to be used for later
	purchases.
\item A user must be able to purchase an artifact, which will then be associated
	with her account, for as long as the artifact is available on the
	service.
\item A user must be able to rent a artifact, which will then be associated
	with her account for 2 days.
\item A user must be able rate any artifact.
\item A user must be able to change her rating of any artifact.
	This cannot be registered as a new rating, but must be an alteration of
	the old one, so that a user can only give the same artifact \emph{one}
	rating.
\item A user must be able to overview artifacts, sorted by the following
	criteria: lexicographic, release date, genre, rating, and
	popularity (per view count).
\item A user must be able to locate a specific artifact through text search,
	listing artifacts where any property matches the search string.
\item A user must be able to watch any artifact associated with her account
	in the browser, as long as it is still available through the service.
\item A user must be able to download any artifact associated with her account,
	as long as it is still available through the service.
\item A user must be able to retrieve a listing of any artifact that is, or has
	been, associated with her account, as long it is still available through
	the service.
\item A user must be able to find all artifacts that are currently associated
	with her account, as long it is still available through the service (I.e.
	artifacts that are available to the user at the given time).
\setcounter{enumTemp}{\theenumi}
\end{enumerate}
\textbf{Managers:}
\begin{enumerate}
\setcounter{enumi}{\theenumTemp}
\item A manager's rights extend those of the user, so that a manager can do
	everything the user can.
\item A manager must be able to upload movies/songs
\item A manager must be able to create and edit the properties of any artifact,
	except for the \emph{extrinsic} properties; average rating and view
	count.
\item A manager must be able to delete movies/songs from the service.
\end{enumerate}

\subsubsection{Security and other Design Decisions}

When deciding on the
requirements for the system, several criteria are considered, and some
requirements which might be crucial in a real-life product are left out. These
decisions are mainly made on the basis that this is an educational project,
making these requirements more or less irrelevant to the project:
\begin{itemize}
\item Our download links stay active at all times, so given a link, 
	anyone can download/view an artifact without authenticating.
\item In a real-life renting service such as iTunes, once a user starts playing
	the movie that she has just rented, the movie will only be available
	for viewing in the next 48 hours. When a user rents and downloads a
	movie or song using our service, they can keep the file for as long as
	they want. However, the file is only available for download for a
	limited period of time.
\item Real-money payment is not implemented. Rather, a "fake" balance where
	people can deposit money to their account, for use when purchasing or
	renting movies and songs. However, given access to an E-commerce
	system, real-money payments could easily be implemented on top of his
	structure.
\end{itemize}

Notably, the web service and its clients have been developed separately.
Therefore, the requirements have become a conjunction of the SMU- and
SWU-clients' requirements, through which the web service's requirements are
implicitly described.

This sort of implicit requirements specification might not be recomendable, but
in this case the requirements were simple enough that it did not cause any real
trouble. However, care had to be taken when determining which requirements to
handle in the client, and which to handle in the API. Surprisingly, the SMU
client/SWU client situation somewhat simplified the issue. Having two
unrelated clients to the service, the web service interface came to be seen as a
general API, rather than a one-off, like the clients.
This view of the web service, made it an obvious choice to place
limiting requirements in the clients, and leave the service as flexible as
possible.

\textbf{For example... req 25}

\subsection{Requirements}
As part of the analysis, we need a set of requirements for our product to
establish what users will and will not be able to do with the system. This
section describes and discusses our choice of requirements and use cases.

We did not immediately arrive at our current requirements. They have changed
over a series of iterations in which we revised and discussed them internally in
the group, as well as with the Singaporean team. For example, our first draft of
requirements did not include songs at all. It was not until the Singaporean team
told us that they were required by their professor to include songs in their
service that this became part of our requirements. The Singaporean team also
made suggestions that media files include ratings and view count to allow us to
sort by rating and popularity. In general, the requirements were almost
exclusively changed on request from the Singaporean team. We, the ITU team,
had few additions to our initial specification.

\label{sec:requirements}
In the following requirements document, the word "user" shall refer to an
existing, registered user. Artifact shall refer to a movie or song. (I.e. the
products available for rent/purchase in the system) There are two types of
primary actors: users and managers. A manager’s rights are a superset of a
regular user's. In addition, managers do not have to purchase anything. 

\textbf{Users:}
\begin{enumerate}
\item A (non) user must be able to register an account.
\item A user must be able to login to her account.
\item A user must be able to edit her account.
\item A user may deposit money into her account, to be used for later
	purchases.
\item A user must be able to overview artifacts, sorted by the following
	criteria: lexicographic, release date, genre, rating, and
	popularity (most viewed/bought).
\item A user must be able to locate a specific artifact through text search,
	listing artifacts where any property matches the search string.
\item A movie must have properties describing its title, year of
	release, director(s), and genre(s).
\item A song must have properties describing its title, year of
	release, artist(s), album title, and genre(s).
\item Additionally, an artifact must have properties to describe its unique id,
	an URI pointing to an image (E.g. cover image for movies), an average
	user-rating, a textual description, a purchase price, a rental price,
	and a view count.
\item A user must be able to purchase an artifact, which will then be associated
	with her account, for as long as the artifact is available on the
	service.
\item A user must be able to rent a artifact, which will then be associated
	with her account for 2 days.
\item A user must be able to watch any artifact associated with her account
	in the browser, as long as it is still available through the service.
\item A user must be able to download any artifact associated with her account,
	as long as it is still available through the service.
\item A user must be able to retrieve a listing of any artifact that is, or has
	been, associated with her account, as long it is still available through
	the service.
\item A user must be able to find all artifacts that are currently associated
	with her account, as long it is still available through the service (I.e.
	artifacts that are available to the user at the given time).
\end{enumerate}
\textbf{Managers:}
\begin{enumerate}
\item A manager's rights extend those of the user, so that a manager can do
	everything the user can, albeit without purchase.
\item A manager must be able to upload movies/songs
\item A manager must be able to create and edit the properties of an artifact,
	except for the \emph{extrinsic} properties; average rating and view
	count.
\item A manager must be able to delete movies/songs from the service.
\end{enumerate}

\subsubsection{Security and other Design Decisions} When deciding on the
requirements for the system, several criteria are considered, and some
requirements which might be crucial in a real-life product are left out. These
decisions are mainly made on the basis that this is an educational project,
making these requirements more or less irrelevant to the project:
\begin{itemize}
\item We decided to leave out user authentication in our web services, so given
	the URL, anyone can download a movie our song without authenticating.
	The data structure does, however, support user authentication with each
	user having a password.
\item In a real-life renting service such as iTunes, once a user starts playing
	the movie that she has just rented, the movie will only be available
	for viewing in the next 48 hours. When a user rents and downloads a
	movie or song using our service, they can keep the file for as long as
	they want. However, the file is only available for download for a
	limited period of time.
\item Naturally, we have not implemented a real payment service but rather a
	"fake" balance where people can deposit money to their account, for use
	when purchasing or renting movies and songs.
\end{itemize}

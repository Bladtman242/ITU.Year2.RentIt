\subsection{Internal interfaces}
In this section we will go through the interfaces between our web service implementation and the business logic of our application. Furthermore we will try to explain the descisions we made and why we made them.

In our backend system we have decided to keep an abstraction layer between our WCF service implementation and the business logic such as persistence, authentication, and data querying. This choice was inspired by the facade design pattern which is known to promote low coupling between application layers. It did also fit our wish to keep a model-view-presenter separation with the web service acting as the presenter, this allows us to have several web services based on the same data model and business logic. This could be handy if the client the SMU students required a different interface for the web service than the client we wanted to create ourselves. We also made the decision in order to keep cohesion high, both in the web service implementation class and the class handling our data, while also keeping coupling low. This is examplified by the fact that we could change the design of the database without having to change anything in web service, or we could have several web service implementations using the DBAccess class without these services having anything to do with each other. As shown in the last example this decision also allowed easier re-use of the methods accessing and editing data.

In order to prevent the DBAccess file from being over a thousand lines long we decided to split it up into several files using the partial class feature. This allowed us to have several documents with more defined areas of expertise making the code easier to look through.  The different files in the class mimic the entities in the database (i.e DBMovie and DBUser).

The DBAccess uses the classes Movie, Song and User to model information stored in the database. When the web service receives instances of these classes, from the DBAccess object it queries, they have to be converted to a format which can be passed as JSON to the consumer of the service. To convert the classes in to their respective wrapper classes (i.e. MovieWrapper) we decided to use extension methods as this allowed us to keep the coupling between the service and the dbaccess low. With extension methods the service itself defines the wrapper classes it uses and how to transform the database classes into these wrappers. This allows different services to return different outputs even though they both use the same DBAccess class.
\subsection{Internal interfaces}
In this section we will go through the interfaces between our web service
implementation and the business logic of our application, and explain the
decisions we made.

In our back-end system we have decided to keep an abstraction layer between our WCF service implementation and features such as persistence, authentication, and data querying. This choice was inspired by the facade design pattern which promotes low coupling between application layers, and fit our wish to keep a model-view-presenter separation with the web service acting as the presenter, as this would allow us to have several web services based on the same data model and business logic.

This could have come in handy if the SMU client required a different interface
for the web service than the client we wanted to create ourselves. We designed
with high cohesion in mind, exemplified by the fact that we could change the
design of the database without having to change anything in the web service, or
we could have several web service implementations using the DBAccess class
without these services having anything to do with each other. As shown in the
last example this decision also allowed easier re-use of the methods accessing
and editing data.

In order to prevent the DBAccess file from being over a thousand lines long we
decided to split it up into several files using the partial class feature. This
allowed us to have several documents with more defined areas of expertise
making the code easier to look through. The different files in the class mimic
the entities in the database (eg. DBMovie and DBUser).

The DBAccess uses the classes Movie, Song and User to model information stored
in the database. When the web service receives instances of these classes, from
the DBAccess object it queries, they have to be converted to a format which can
be passed as JSON to the consumer of the service. To convert the classes in to
their respective wrapper classes (i.e. MovieWrapper) we used C\# extension methods
as this allowed us to ensure low coupling between the service and the DBAccess.
With extension methods the service itself defines the wrapper classes it uses
and how to transform the database classes into these wrappers. This allows
different services to return different outputs even though they both use the
same DBAccess class.

\subsection{Evaluation of the API}
\label{sec:evalapi}

Developing the client was a great opportunity to use and evaluate the web
service. In general, using javascript in the client simplified
interoperability, as all conversion between data-types was handled in the
language. As such we could use the user object returned when logging in
directly as the user object in the client.

When getting big amounts of data (all movies, all songs, all users, etc.) it
was not very fast, with the current API, to go through these.

For example, when displaying a single movie to a user that
is logged in, the movie is first retrieved. This is straight-forward, and returns only the required
data. To check whether the movie should be watchable and downloadable, however, we had to check whether
it was in the set of movies currently available to the user. This required getting a lot of data that
wasn't necessarily required.

As transferring data over the network is the obvious bottleneck in our system (the actual computations
on either side are very quick), sending as little data as possible is desirable. We could have solved this
problem by having a call specifically for checking whether a user owns a specific artifact.

Additionally, when getting lots of data (for example when showing all movies) only a little of the data
is actually used at first (showing the highest rated movies, and the first 12 movies when sorting
alphabetically). We could have decreased the impact of the bottleneck in this case by introducing
pagination: allowing the client to get a few elements at a time, calling the API again when more elements
need to be loaded.

With the current API, to upload a movie, it is required to first call \verb+movies/upload+ and then
\verb+movies/create+. This largely originated as a technical constraint because JSON is not an appropriate
format for binary data (like movie files). In the client this was, fortunately, easily circumvented by
allowing file upload to happen asynchronously and automatically carrying over the required information
from the \verb+movies/upload+ call to the \verb+movies/create+ call. The same was the case for the creation
of new songs.

In conclusion, several things could be improved with the API, but it is
currently a working and easily learnable interface.

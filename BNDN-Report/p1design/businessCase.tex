\section{Business Case}
In this section we will give a brief introduction to our business case, in order to introduce the reader to the developed system.

\subsection{Business opportunity}
With this project we have wanted to address the problem of piracy in the media industry. Piracy is an often committed illegal activity, with recent research showing that around 30\% of 15-29 year-olds (in Denmark) engage in piracy of movies and tv-shows\cite{pirates}. This same study showed that most pirates would prefer to acquire their media in legal ways, but there is a lack of options with the same high accessibility, which points them in the illegal direction.

Based on the study mentioned above, we believe that we can adress the problem of piracy by providing an accessible way for customers to acquire media in legal ways.

\subsection{Concept elaborated}
We aim to provide a service, which allows users to buy and rent movies and songs. Upon purchase, the user will be able to both stream and download the media directly from our service, as \textbf{accessiblity is a key quality in the service}.

We hope to convert media pirates, who want to acquire media in a legal manner, but simply lack the means to do so. This is to be achieved through accessible, DRM-free media. Since media streaming is popular\cite{ott} we do not wish to restrict ourselves to this segment, as we believe a typical  family will have interest in our service too.

While big brands already exist in the media streaming industry (e.g. NetFlix for movies/series, or Spotify for music) our concept differs in certain ways, as detailed below:

\begin{itemize}
    \item We keep both movies and media gathered in one single service. We believe that, 
        in the upcoming era of smart tvs and media centers \cite{smarttv}, it will be
        desireable to have media stored in one application rather than having several
        different apps on a single media center.
    \item We are do not use the \emph{Recurring Revenue model (Subscription model)}\cite{businessmodel}. Our users are not limited by a subscription peroid in which they can access their media. Instead, a bought item will be treated as their property, free for download or streaming at any time, thus providing high accessiblity. In addition, recent research\cite{ott} has predicted that subscription services will not dominate in the future, while other forms of digital content distribution will.
\end{itemize}
    
\subsection{Impacts and risks}
This section will elaborate on the immediate risks associated with the concept. Since
streaming services hav been around for several years the technology needed has matured,
and as such we have few risks associated with the technology aspects. The technology
risk with the largest impact on the business, is that we cannot find a fullfilling
security mechanism for the downloaded files, making them shareable without copyright concerns. Note that
we have found feasibility tests of this mechanism to be outside of the scope of this
report.

The other risks, with large impact, are external factors:
\begin{itemize}
\item A solid customer base may not be achieved.
\item A partnership with content providers may not be achieved.
\end{itemize}

These risks are mostly concerned with marketing, and strategies to avoid them is considered out of scope for this report.

\subsection{Deciding on the External API Form}
The two teams quickly agreed on using a RESTful\footnote{REpresentational State
Transfer, a type of API based on the Hyper-Text Transfer Protocol (HTTP),
http://www.oracle.com/technetwork/articles/javase/index-137171.html}
JSON-based\footnote{ JavaScript Object Notation, a lightweight data-interchange
format based on a subset of the JavaScript programming language,
http://www.json.org/} API as the interface between frontend (client) and
backend (server). This still left a lot to design.

When designing an API it is important for it to be intuitive and attractive, as this
allows third-party developers, as well as our overseas colleagues, to use the API for
their purposes. Third-party developers in general represent a huge business opportunity
in our case, as the system requires purchases in order to actually access artifacts. This
means that third-party clients may create new revenue sources for us.

When discussing how to best attack the problem of what a \emph{nice API} is, we found two
options.

Either we could let the API handle arbitrary data types or have specific services for
every allowed data type. The difference between the two possibilities can be illustrated
through which services are offered by them:

\begin{table}[hbt]
    \centering
    \begin{tabular}{ | l | l | l | }
        \hline
        \textbf{Abstraction level} & High abstraction & Low abstraction \\
        \hline
        \textbf{Services provided} & files/ & movies/ \\
        &  & songs/ \\
        \hline
    \end{tabular}
    \caption{Services offered by different abstraction levels}
\end{table}

In the high abstraction case a call to the files service may return any of the
supported data types, so the client can not know the return type for sure at
the time of calling the service. This problem can be solved in the following
ways: first off, the client discovers content through a service that will also
disclose the type of the content (so it is already aware of the type) and,
secondly, the files-service itself will return the type of the content as well
as the content itself.

In the low abstraction case, the movies- and songs-services will always return
respectively movies and songs.

This difference can be illustrated by looking at the services as if they were method calls.
In this case, the files-service would have a return-type of "Object", whereas the movies-
and songs-services would have "Movie" and "Song".

The high abstraction level works like a dynamically typed language (PHP, Javascript) would
work, whereas the low abstraction level works like a type-strong language like C\# or Java.

\subsubsection{Cost of Change}
In the high abstraction level service it is easy to expand the range of data types we offer,
as they will be using the same API. This can with great advantage be paired with a discovery
service that lets clients discover the supported data-types in real-time. The cost of change
is pretty much non-existent.

This has \emph{a lot} of coolness factor.

This approach is very dynamic and allows for updates to be rolled out seamlessly. It is, however,
best suited for a tuple-space based database and not a relational one, such as the one we have
chosen to work with.

In a relational database, using the high abstraction would require either
massive scans of one huge table containing all files in the system or having
multiple indexes for the same, huge table, which would be quicker, but also
take up double the space.

The low abstraction level is very well suited for the relational database.
Creating a table for each of the services (one for movies, one for songs) is a
natural choice and lets us have the desired attributes for the types.
Additionally searching is easier and fewer indexes are needed per table (so it
takes up less space).

On the downside, the low abstraction level requires for entire new tables and services to be added
whenever we want to support a new data type. In turn, adding new services to the API does not make
them accessible from the clients. In order to fully roll out an update, all clients,
including third-party ones, must be updated. This all adds up, so the cost of change is massive.

\subsubsection{Final decision}
In our application we consider the support of a new data type to be somewhat a
rare occasion. Our business is centered around a very specific set of products -
movies and songs - and it is with this line of products it will be known. Add to
that the fact that relational databases are extremely common, while tuple
space database implementations are sparse, and even more so when opting for C\#
compatibility.

With this in mind, a high cost of change is considered acceptable. With the advantages we trade in for
it, it is a more than acceptable potential loss, so we decided on a low abstraction API, and the relational
data model that goes with it.

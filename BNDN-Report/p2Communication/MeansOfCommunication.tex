\subsection{Means of Communication}

Throughout the project we had a strong focus on continuously optimizing our
means of communication. We focused on the communication being clear and
responsive, but also comfortable. We wanted to avoid confusion where possible,
but many issues required internal discussions in the teams e.g. before we could agree 
on changes to the API we had to assert the cost of change on our side, whether it
was worth it, and whether there were any readily available and more efficient
alternatives. Thus, avoiding confusion was not a simple task.

The terms of the first meeting were beyond our influence. The format was a
video conference, and being able to see and hear the
other team in real-time gave some insight into when they were thinking, when
they were getting ready to speak, etc. Despite the sound-quality being sub-par,
this was a great way of communicating - clear, quick, and to the point.

For our second meeting we chose the technology resembling the video conference
the most. The choice landed on Skype\footnote{ Skype is, among other things, a
Voice over Internet Protocol (VoIP) service, allowing participants to
communicate with video and sound}, which has
many of the same features.

To our disappointment Skype proved highly unreliable when using the internet-connection of the ITU: the video was blurry
and the sound lagging behind. We had a choice between giving up Skype and giving up sitting together as a team when communicating.

We decided to give Skype another chance and moved to our respective residences to call from there. This increased the quality of our communication. With group calls it was
possible for the Danish team to first meet (virtually) to discuss the agenda, before calling up the Singaporean team. We had to give
up video for this (as it turns out to be a paid feature of Skype to have video in group-calls), but all-in-all it was an improvement.

\subsubsection{The change to textual communication}
Skype was mainly used in the beginning of the project, when we were trying to reach an agreement on API design. It was very
useful to have several calls per week where we could introduce, propose, and discuss new features. Some items were
agreed upon instantly, while others were put on the agenda for the next meeting, allowing the teams to thoroughly discuss the feature
and all that it might entail.

As the changes to the API became smaller and more irregular, we moved to a more text-based form of communication. We had
previously been using e-mail as a way to prepare for meetings and discuss agendas, but as we had less to discuss at the
meetings this became obsolete.

Communicating via Facebook\footnote{Facebook is a social network on which written group communication is well integrated} came as a natural solution as, in a young generation such as ours, the majority of people
already use the service. It was quick to set up a group and use this as a form of general communication between the teams. This way
of communicating preserved the advantage that teams could get together and discuss changes before reporting back,
but we lost the "human factor" by no longer being able to see and hear each other.
Not being able to see each other removes the factors of intonation and facial expressions in the communications increasing the risk of
misunderstandings , but we won a lot of time and efficiency.

This turned out to be a productive change as communication became less formal and more to the point.  

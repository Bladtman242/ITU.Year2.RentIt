\subsection{Communicating with the Singaporean team}
\label{sec:communicating}
%rephrase intro It is commonly understood that different cultures have
%different norms Less widely known, but well documented, are the difficulties
%the cultural differences impose on inter-cultural collaboration

Immediately at the first meeting with the Singaporean team, the first hint of
cultural differences presented itself in the form of Hofstede's first dimension
of culture, adherence to hierarchical organization structures\cite{surprises}.
The Singaporean team wanted to know who the team leader of Danish team was.
This was something the Danish team had not discussed, or even considered, as
they were used to acting in small group democracies (i.e. with no leader). The
project groups the Danish team are used to to participate in are small enough
that everyone can be quite intimate with the project in its entirety. Arguably
eliminating the need of government. In fact, the Danish contact person was
chosen only shortly before the meeting, and with an air of necessity rather
than desire.

For a while, the cultural differences seemed negligible, but as the project
progressed communication between the two teams became increasingly troublesome.
More than once it got to the point where one Danish team member felt it necessary
to put down his foot\cite{enough}. The Danish team members perceived a tendency in the Singaporean
team; the Singaporean team members would seemingly agree to proposals from the Danish team, but
would then turn on it later. This tendency is illustrated by a couple of
examples:

During the first meeting, a second meeting was arranged for two days later. Not
long after, the Singaporean team announced that they would not be able to make the
agreed appointment, and proposed to meet the following Saturday instead.
Despites the verbally unambiguous agreement however, the Danish team's
representatives held Saturdays meeting alone. [Reference needed]

Another example of apparent agreement: Shortly after having mutually expressed
agreement on the API of the web service (and thereby the supported requirements
for the Danish client) the Singaporean team started requesting API support for features
already described in the shared documents. [reference needed] Indicated that
the Singaporeans either had not read, or not been able to understand, the technical
specification document supplied by the Danish team.

Approaching the Singaporean team's deadline, the Danes had their easter
holidays. This wasn't exactly a surprise, and the Danish team had tried, and
thought they had succeeded, to cement the service-requirements so only actual
implementation work, demanding little to no communication, was left for the
holidays. The Singaporeans however, had planned differently. Due to having deadlines
for other projects early in the process, they had delayed finalizing their
requirements, and announced most of their change requests in this period. This
resulted in great stress for both the Danish team, who had made plans for the
holidays, and for the Singaporean, who became increasingly anxious they might not
make their deadline.

The possibility that these disagreements are unrelated to cultural differences
between the teams, and hence are products of dislike or indifference must be
explored: The Danish team agrees that the Singaporean team's emails, Facebook messages,
and appearance during meetings generally showed a positive demeanor. That, and
the explicitly expressed wish to collaborate with the Danish team, even when
facing disagreement makes this explanation unlikely. (see appendix
\ref{app:robertsletter})

Let us explore the possibility of cultural differences acting as catalyst for
misunderstanding. In some cultures, (commonly in Asia) public disagreement is
somewhat stigmatized (Hall's fifth dimension of culture\cite{surprises})
\cite{herbsiemens}, and rather than explicitly saying \emph{"no"} the Singaporean team
might have, subtly, as perceived by western standards, hinted a reluctance that
the other team was expected to, but did not know to, react on.


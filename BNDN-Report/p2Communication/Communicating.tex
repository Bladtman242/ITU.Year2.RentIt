\subsection{Communicating with the Singaporean team}
\label{sec:communicating}

At the first meeting with the Singaporean team, the first hint of cultural
difference presented itself in the form of Hofstede's first dimension of
culture: adherence to hierarchical organization structures\cite{surprises}.
This was exemplified by the Singaporean team wanting to know who the Danish
team leader was, the Danish team had not discussed, or even considered, as we
were used to acting in small group democracies. The project groups the Danish
team were used to participate in are small enough that everyone can be quite
intimate with the project in its entirety, which, arguably, eliminates the need
of government. In fact, the Danish contact person was chosen only shortly
before the meeting, and with an air of necessity rather than desire.

For a while, the cultural differences seemed negligible, but as the project
progressed communication between the two teams became increasingly troublesome.
More than once it got to the point where the Danish team felt it necessary to
put a foot down (see Appendix \ref{app:enough}). The Danish team members perceived a tendency
in the Singaporean team to agree to
proposals and deadlines from the Danish team, but then change their stance when
approaching deadlines. This tendency is illustrated by a couple of examples:

During the first meeting, a second meeting was arranged for two days later. Not
long after, the Singaporean team announced that they would not be able to make
the agreed appointment, and proposed to meet the following Saturday instead.
Despite the verbally unambiguous agreement, the Danish team's
representatives held Saturdays meeting alone (see Appendix \ref{app:meetlogs},
Minute 2).

Shortly after having mutually expressed
agreement on the API of the web service (and thereby the supported requirements
for the Danish client) the Singaporean team started requesting API support for
features already described in the shared documents.
This indicated that the Singaporeans either had not read, or not been able to
understand, the technical specification document supplied by the Danish team.

Approaching the Singaporean team's deadline, the Danes had their easter
holidays. This was not exactly a surprise, and the Danish team had attempted,
and thought they had succeeded, to cement the service-requirements so only
actual implementation work was left for the holidays, demanding little to no
cross-team communication. The Singaporeans however, had planned differently.
Due to having deadlines for other projects early in the process, they had
delayed finalizing their requirements, and announced most of their change
requests in this period.  This resulted in great stress for both the Danish
team, who had made plans for the holidays, and for the Singaporeans, who became
increasingly anxious they might not make their deadline.

A feature that the Singaporean team wanted implemented in the last week was
a rating feature. However the Danish team was not happy spending time on this
feature as it did not seem key to their project and we were already pressed on 
time. To trumph the feature through the Singaporean team let it us know
that it was a requirement from their professor. It later turned out that the
professor had not required a rating feature and his name was only used
to pressure us into creating the feature. This left us angry with the 
Singaporean team as they had manipulated us, but as the collaboration
segment of the project was nearing the end we let it rest.

The possibility that these disagreements are unrelated to cultural differences
between the teams, and hence are products of dislike or indifference must be
explored: The Danish team agreed that the Singaporean team's emails, Facebook
messages, and appearance during meetings generally showed a positive demeanor.
That, and the explicitly expressed wish to collaborate with the Danish team,
even when facing disagreement makes this explanation unlikely (see Appendix
\ref{app:robertsletter}).

Let us explore the possibility of cultural differences acting as catalyst for
misunderstanding. In some cultures, (commonly in Asia) public disagreement is
somewhat stigmatized (Hall's fifth dimension of
culture)\cite{surprises}\cite{herbsiemens}, and rather than explicitly saying
\emph{"no"} the Singaporean team might have, subtly, as perceived by western
standards, hinted a reluctance that the other team was expected to, but did not
know to, react on. The difference in the two teams' adherence to leadership is
similar to that described in \emph{Global Software Development at
Siemens}\cite{herbsiemens}. Notably, in this project the difference was between
having a leader and not having a leader, perhaps making the effects larger than
experienced in the Siemens project, where the difference was the degree to which
leaders micromanaged.

In some cultures (commonly in Asia), public disagreement is somewhat
stigmatized (\emph{Hall's fifth dimension of
culture})\cite{surprises}\cite{herbsiemens}, and rather than explicitly saying
\emph{no} the Singaporean team might have hinted a reluctance that the other
team was expected to react on, but the Danish team did not notice this, as this
kind of hinting (as opposed to speaking ones mind) is uncommon in Danish
culture. The difference in the two teams' adherence to leadership is similar to
that described in \cite{herbsiemens}. In our project the difference was between
having a leader and not having a leader, perhaps making the effects larger than
experienced in the Siemens project, where the difference was in the degree of
the leaders' micromanagement.

Some miscommunication appears to be the result of unclear communication with
respect to planning. The easter situation corresponds well to that described in
\cite[sec.~3.1.2]{herbsiemens}, where the lack of a commonly agreed-on plan
intensifies the demands on an already insufficient communication channel.

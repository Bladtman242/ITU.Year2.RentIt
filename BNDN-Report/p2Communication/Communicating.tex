\subsection{Communicating with Singapore - A clash of cultures}
\label{sec:communicating}
%rephrase intro
%It is commonly understood that different cultures have different norms.
%Less widely known, but well documented, are the difficulties the cultural differences impose on inter-cultural collaboration.

Immediately at the first meeting with the SMU team, the  first hint of cultural
differences presented itself in the form of Hofstede's first
dimension\cite{surprises}. The SMU team wanted to know who the team leader of SWU team was.
This was something the SWU team had not discussed, or even considered, as they were used to acting in small group democracies.

At first, the differences went unnoticed, but as the project progressed
communication between the to teams became increasingly difficult. More than
once it got to the point where one SWU team member felt it necessary to put
down his foot.\cite{enough}
The SWU'ers perceived a tendency in the SMU team; the SMU'ers would seemingly agree to proposals from the SWU team, but would then turn on it later.
This tendency is illustrated by the following eamples:

During the first meeting, a second meeting was arranged for two
days later. Not long after, the SMU team announced that they would
not be able to meet the agreed appointment, and proposed to meet 
the following Saturday instead.

Shortly after having mutually agreed on the API of the web
service (and thereby the supported requirements for the SWU client)
the SMU team startet requesting API support for features already
described in the shared documents. [reference needed]

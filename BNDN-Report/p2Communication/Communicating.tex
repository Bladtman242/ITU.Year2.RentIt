\subsection{Communicating with Singapore - Agree to disagree}
\label{sec:communicating}
%rephrase intro
%It is commonly understood that different cultures have different norms
%Less widely known, but well documented, are the difficulties the cultural differences impose on inter-cultural collaboration

Immediately at the first meeting with the SMU team, the  first hint of cultural
differences presented itself in the form of Hofstede's first
dimension of culture\cite{surprises}. The SMU team wanted to know who the team leader of SWU team was.
This was something the SWU team had not discussed, or even considered, as they
were used to acting in small group democracies.

At first, the differences went unnoticed, but as the project progressed
communication between the two teams became increasingly troublesome.
More than once it got to the point where one SWU team member felt it necessary
to put down his foot\cite{enough}.
The SWU'ers perceived a tendency in the SMU team; the SMU'ers would seemingly
agree to proposals from the SWU team, but would then turn on it later.
This tendency is illustrated by a couple of examples:

During the first meeting, a second meeting was arranged for two
days later. Not long after, the SMU team announced that they would
not be able to make the agreed appointment, and proposed to meet the following
Saturday instead. Despites the verbally unambiguous agreement however,
the SWU team's representatives held Saturdays meeting alone. [Reference needed]

Another example of apparent agreement:
Shortly after having mutually agreed on the API of the web
service (and thereby the supported requirements for the SWU client)
the SMU team started requesting API support for features already
described in the shared documents. [reference needed

The possibility that these disagreements are unrelated to cultural
differences between the teams, and hence are products of dislike or
indifference must be explored: The SWU team agrees that the SMU team's emails, Facebook
messages, and appearance during meetings showed a positive demeanor. That, and
the explicitly expressed wish to collaborate with the SWU team, makes this
explanation unlikely. (see \ref{sec:robertsletter})

Let us explore the possibility of cultural differences acting as catalyst for
misunderstanding. In some cultures, (commonly in Asia) public
disagreement is somewhat stigmatized (Hall's fifth dimension of
culture\cite{surprises}) \cite{herbsiemens}, and rather
than explicitly saying \emph{"no"} the SMU team might have, subtly,
as perceived by western standards, hinted a reluctance
that the other team was then expected to react on.


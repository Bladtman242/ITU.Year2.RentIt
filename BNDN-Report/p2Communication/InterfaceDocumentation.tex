\subsection{Interface Documentation}

In the beginning of the project the two teams had not agreed on how to share documentation - each of the
teams were left to their own devices.

In the Danish team we used a combination of well-documented code and various diagrams to show the relationship
between classes in the program, entities in the backend, and so on. As the project is fairly limited in scope,
all members of the Danish team were aware of the more abstract ideas of the project, so this documentation was
sufficient.

Additionally, comments for the top layer, found in the \verb+JsonServices+ project, were deferred as this part
of the project was subject to the most change and, additionally, was mostly self-documenting in its format.

%Begin example of C# API interface implementation
\begin{table}[h]
    \begin{tabular}{ | l | }
        \hline
        [WebInvoke(Method = "POST", \\
        \hspace{5em}
            RequestFormat = WebMessageFormat.Json, \\
        \hspace{5em}
            ResponseFormat = WebMessageFormat.Json, \\
        \hspace{5em}
            BodyStyle = WebMessageBodyStyle.WrappedRequest, \\
        \hspace{5em}
            UriTemplate = "\{id\}/purchase")] \\
        SuccessFlag PurchaseSong(string id, int userId); \\
        \hline
    \end{tabular}
    \caption{Example of the interface implementation (ISongService) of the service's action to purchase
        a song. As is seen, the code is rather verbose, in that it details request method, request- and response-format,
        as well as the actual URI used.}
\end{table}
%End example

Adding documentation comments to this code would simply require for the team to learn a \emph{different} syntax.

With requests for changes coming in rapidly, some asking for features that had already been implemented, we
realized that it would be more optimal to share updated documentation with the Singaporean team. We had previously
shared the first draft with them in the context of agreeing on initial requirements for the API, and it was
straight-forward to move this to Google Docs\footnote{Google Docs allows users to share and collaboratively edit
documents online, http://docs.google.com/} and bring it up to date. Going forward this was our single
reference-point for updated documentation of the API.

\subsubsection{Format of Documentation}

After our first discussion with the Singaporean team, the Danish team had the impression that they were
relatively technically savvy: they had chosen to use a technology that was not the standard choice, and
they had researched what this would require on our part.

As JSON is a relatively easily understandable format, and as properly designed RESTful APIs should be
understandable from just the URI, we decided to use the two in combination to document it: a HTTP-method
and URI heading with parameters denoted in curly-brackets, accompanied by an optional JSON-format request
body as well as a required response body, also in JSON, which denotes the return format if the request was
a success.

%Following is an example of the JSON documentation
\begin{table}[h]
    \begin{tabular}{ | l | }
        \hline
        POST users/create \\
        \textbf{\{ name: /*string*/, username: /*string*/, email: /*string*/, password: /*string*/ \}} \\
        \{ success: /*boolean*/, message: /*string*/ \} \\
        \hline
    \end{tabular}
    \caption{Example of API documentation showing the request and response formats for creating a new user.}
\end{table}
%End of example

%NOTICE CITATION NEEDED BELOW!

In accordance with proper RESTful design, requests that fail will return the proper error codes[citation
needed] in the header of the HTTP-response.

As we had proposed this format to the Singaporean team, and they had agreed to it, this seemed like an
acceptable solution. In addition to this, we expected them to be fairly aware of how the JSON format works,
as they had proposed it themselves.

\subsubsection{Issues with Shared Documentation}

We did not encounter any massive issues with our way of sharing the documentation of the interface.

With a shared document where all members of both teams had the rights to edit, one could imagine a situation
where changes were made to the documentation without being made to the actual implementation, causing a
discrepancy that might take a long time to notice and fix. Such a situation did, however, not occur.

We encountered only one issue with the whole of the documentation process, which we did not find a solution for
during the time of the project.

On several occasions, when the Singaporean team requested features, the features turned out to already exist, be
implemented and fully documented. This was a case of simply not reading the documentation before posting an idea.

This led to a growing amount of time spent checking that the feature was indeed documented and implemented, and then
informing the Singaporean team of the situation. Mostly this was a minor issue, but adding up, as it did, it led to
an increasing level of annoyance, which was not healthy for collaboration.
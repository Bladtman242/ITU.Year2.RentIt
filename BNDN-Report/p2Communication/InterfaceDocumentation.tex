\subsection{Interface Documentation}

% stuff about where the documentation is placed (deciding towards a shared resource)

\subsubsection{Format of Documentation}

After our first discussion with the Singaporean team, the Danish team had the impression that they were
relatively technically savvy: they had chosen to use a technology that was not the standard choice, and
they had researched what this would require on our part.

As JSON is a relatively easily understandable format, and as properly designed RESTful APIs should be
understandable from just the URI, we decided to use the two in combination to document it: a URI heading
with parameters denoted in curly-brackets, accompanied by an optional JSON-format request body as well as
a required response body, also in JSON, which denotes the return format if the request was a success.

% Example of JSON API documentation

In accordance with proper RESTful design, requests that fail will return the proper error codes[citation
needed] in the header of the HTTP-response.

As we had proposed this format to the Singaporean team, and they had agreed to it, this seemed like an
acceptable solution.

%how it all went wrong when they didn't read it.
Even before the first lecture in the \emph{Software
Development in Large Teams with International Collaboration} course, rumors
about the Singaporean university were circulating. As the only source of
information about what to expect from working with a Singaporean team, the
rumors were taken to heart and very much shaped the expectations of the team.

One team, which had been through the same project the previous year, was telling
the story of how they had, in
practice, been doing both the work they were supposed to do and the work the
Singaporean team was supposed to do. This was an outlier in terms of extremity,
but most rumors confirmed that expectations should be kept low, with respect to
the Singaporeans' efforts.

It was with these expectations we went into the video conference room for the
first virtual meeting with the Singaporean partners, that we were to work with
for the following weeks. During this first meeting, however, all of our
expectations were proven wrong: the Singaporean team was well prepared, and we
ended up going by their agenda as they had covered all that we planned to talk
about and more.

What made the greatest impression on us was that the Singaporean team had
already set their sights on a technology and researched exactly what this would
mean for our common interface. They wished to develop using AngularJS, and
inquired as to whether it would be possible for us to implement a RESTful
JSON-based interface on the server.

Leaving that initial meeting our expectations were through the roof: despite
everything we had heard it seemed that we had been paired with a professional,
dedicated, and well-prepared group.

In the following our relationship with the Singaporean team is elaborated.

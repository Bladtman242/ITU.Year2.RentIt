\section{Introduction}

Through the course, Software Development in Large Teams with International Collaboration,
students at the IT-University of Copenhagen (ITU) are tasked with creating a digital rental
system in collaboration with students from the Singapore Management University (SMU).

The team of students from the ITU (the Danish team) consists of five people, whereas the
Singaporean team consists of three.

The authors of this report are Niclas Benjamin Tollstorff (nben@itu.dk), Niels Liljedal
Christensen (nlch@itu.dk), Sigurt Bladt Dinesen (sidi@itu.dk, Kristian Brink Gansted
(kbri@itu.dk), and Niels Roesen Abildgaard (nroe@itu.dk).

The Singaporean team is made up of Chai Ching Hsiang Robert, Tan Kah How Kelvin, and
Chong Wen Xiong Nick.

The digital rental system that is to be created should consist of
\begin{itemize}
    \item a server program published through an IIS web server, developed by the Danish team;
    \item a client program, using the server, developed by the Singaporean team; and
    \item a client program, using the server, developed by the Danish team.
\end{itemize}

This report concerns our experiences with developing such an application, as well as how
transnational development affected the process. As these two aspects of the development
are easily discussed separately, the report is split in two parts.

In the first part we discuss the process of designing the software as well as the final
result. We touch on alternatives that could have been used and why we – in this particular
case – designed it as we did. We touch upon the design of requirements, use cases, and APIs
as well as the underlying data model.

In the second part we concentrate on the actual collaboration with the Singaporean team,
difficulties faced in general Global Software Development, and how these manifested in
our work. We explore different technologies for conducting virtual meetings, and chronicle
our experiences with shared documentation.

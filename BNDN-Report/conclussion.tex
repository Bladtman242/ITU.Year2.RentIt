\section{Conclusion}
Finishing the project with all central requirements met (refer sections
\ref{sec:requirements} and \ref{sec:evalapi}), we feel satisfied with the product
we have produced. 
The RESTful WCF service turned out to be easy to work with,
both for us and the Singaporeans. Our data model could be a bit more flexible, but
with the scope of the project and the nature of our system, we think it is more than good
enough to fulfil its purpose.
The client was a bit of an experiment, trying out Bootstrap and no server side code,
but even though we had to leave a few requirements unfulfilled, we think it 
turned out very well. We are especially happy with the look and feel, and how well it works with our web service.

Working with the team from Singapore turned out to be a challenge, but for 
different reason than we expected. We learned the hard way that 
cultural differences can sneak up on you if you do not stay alert. After the first meeting with the Singaporeans we
thought we had it all figured out because they
seemed so similar to us, but we overlooked some clear indications that our communication
was hurting.
Fortunately, there were no problems that we could not work out in the end, but
despite working out the problems that occurred, the time spent working them out
could have been used on other parts of the project. 
Had the issues never risen it could have resulted in more, if not all, of our requirements being fulfilled.

Overall, this project has taught us that it is important to stay focused on maintaining good 
communication, and to stay aware of cultural differences that may 
appear at any time throughout a project involving international collaboration.

